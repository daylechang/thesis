At a minimum, only two input files are required to run a basic simulation: the grid file and an OVERFLOW input file. Additional inputs are needed for restarts (restart file) and for tracking the flow solution (FOMOCO input file). 

The flow is initialized to the conditions described in the input file. The \hl{minimum required} is the freestream Mach number, angle-of-attack, side-slip angle, and freestream temperature.



% \section{Visualization}
% 
% Tecplot is a CFD visualization and analysis tool for large data sets. It is capable of loading in the % OVERFLOW PLOT3D grid file format and can calculate important flow parameters such as density magnitude % gradient and plot by iso-surfaces. Tecplot also extracts slices of data, allowing line-plotting of the % important quantities. \cite{Tecplot2017}

% 1. Suhs, N.E., Rogers, S.E., and Dietz, W.E.,”PEGASUS 5: An Automated Pre-Processor for Overset-Grid CFD,” % AIAA-2002-3186, June 2002.
% 2. Noack, R., “SUGGAR: A General Capability for Moving Body Overset Grid Assembly,” AIAA-2005-5118, Jun. % 2005.
% 3. Chan, W.M., and Buning, P.G., “User's Manual for FOMOCO Utilities – Force and Moment Computation Tools % for Overset Grids,” NASA TM 110408.
% 4. Boger, D., and Dreyer, J., “Prediction of Hydrodynamic Forces and Moments for Underwater Vehicles Using % Overset Grids,” AIAA-2006-1148, Jan., 2006.
% 5. Pandya, S.A., Venkateswaran, S., and Pulliam, T.H., “Implementation of Preconditioned Dual-Time % Procedures in OVERFLOW”, AIAA-2003-0072, Jan. 2003.
% 6. Potsdam, M.A., Sankaran, V., and Pandya, S.A., “Unsteady Low Mach Preconditioning with Application to % Rotorcraft Flows,” AIAA-2007-4473, June 2007.
% 7. Baldwin, B.S., and Lomax, H., “Thin Layer Approximation and Algebraic Model for Separated Turbulent % Flows,” AIAA-78-0257, Jan. 1978.
% 8. Baldwin, B.S., and Barth, T.J., “A One-Equation Turbulence Transport Model for High Reynolds Number Wall% -Bounded Flows,” AIAA-91-0610, Jan. 1991.
% 9. Spalart, P.R., and Allmaras, S.R., “A One-Equation Turbulence Model for Aerodynamic Flows,” AIAA-92-% 0439, Jan. 1992.
% 10. Wilcox, D.C., “Reassessment of the Scale-Determining Equation for advanced Turbulence Models, AIAA % Journal, Vol. 26, No. 11, 1988, pp. 1299-1310.
% 11. Menter, F.R., and Rumsey, C.L., “Assessment of Two-Equation Turbulence Models for Transonic Flows,” % AIAA-94-2343, June 1994.
% 12. Degani, D., and Schiff, L.B., “Computation of Supersonic Viscous Flows Around Pointed Bodies at Large % Incidence,” AIAA-83-0034, Jan. 1983.
% 13. Nichols, R.H., “Algorithm and Turbulence Model Requirements for Simulating Vortical Flows,” AIAA-2008-% 0337, Jan. 2008.
% 14. Suzen, Y.B., and Hoffmann, K.A., “Investigation of Supersonic Jet Exhaust Flow by One- and Two-% Equation Turbulence Models,” AIAA-98-0322, Jan. 1998.
% 15. Abdol-Hamid, K., Pao, S., Massey, S., and Elmiligui, A., “Temperature Corrected Turbulence Model for % High Temperature Jet Flow,” AIAA-2003-4070, June 2003.
% 16. Spalart, P.R., Deck, S., Shur, M.L., Squires, K.D., Strelets, M.K., and Travin, A., "A New Version of % Detached-Eddy Simulation, Resistant to Ambiguous Grid Densities," Theor. Comput. Fluid Dyn., Vol. 20, % 2006, pp. 181-195.
% 17. Nichols, R., “Comparison of Hybrid RANS/LES Turbulence Models for a Circular Cylinder and a Cavity,” % AIAA Journal, Vol. 44, No. 6, June 2006, pp. 1207-1219.
% 18. Sickles, W.L., and Steinle, F.W., Jr., “Global Wall Interference Correction and Control for the NTWC % Transonic Test Section,” AIAA-97-0095, Jan. 1997.


% 