\section{Verification}

In order to make the simulation as realistic as possible, the problem was initially set up to match the experiments of Willis et al. and to match the computational simulation performed by Slater. 

In order to verify the simulation, the incoming boundary layer profile was matched with the experiment.

** taken from davis, vyals, and slater
The similarity of the curves for the different Mach numbers led Davis (one effort) and Slater (another effort) to focus on the 90 degree data of Willis et all. Both investigators took the approach of normalizing the bleed plenum pressure by the local surface static pressure, but Davis also included a coefficient to account for the slight overpressure of the bleed plenum at zero flow rates. 

Slater assumed that for the 90 degree hole case, the total pressure in the hole was approximately equal to the local surface static pressure. Davis established a purely empirical scaling based on the extrapolated choked value of the bleed plate.

Davis is equation is this
Slater equation is this



**

Drawing on the works of slater, 

Firstly, the boundary layer