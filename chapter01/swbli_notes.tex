\chapter{Background and Theory} \label{background and theory}

A background on the research done on SWBLI and more specifically on active bleed control will be discussed. The associated experimental and computational results will follow. The fundamentals of CFD will be discussed including a derivation of the Navier-Stokes equations. The appliation of CFD will be discussed in grid generation.
% ----
The separated region and the shock wave system that develops upstream of the separation line oscillate at low frequency, at least two orders of magnitude lower than the energetic scales present in the upstream boundary layer. The origin of these low frequencies is not totally understood, and several models have been suggested to explain their development. A major problem is to separate the low-frequency shock motions, which appear when the flow is separating, from the motions related to unsteady conditions of the upstream boundary layer. This type of unsteadiness is typically based on upstream energetic scales and generate some corrugations of the shock wave (Debieve \& Lacharme 1985; Wu \& Miles 2001; Garnier \& Sagaut ` 2002). The associated frequency scales differ by two orders of magnitude from the low frequencies of the shock motions. \cite{Piponniau2009}
% ----
For decades, the interaction with an incident shock and the compression ramp have been the two most documented cases (see Delery \& Marvin 1986). A recent review of the main properties of these flows can be found in Dolling (2001) \cite{Piponniau2009}

do the shock
movements influence the instantaneous position of the separated region through some
upstream perturbations, or does the unsteadiness of the separated bubble impose the
large motions of the shock?

determined that the main source of low-frequency unsteadiness in shock-induced separated flows seems clearly to be the dynamics of the separted bubble, and small variations in the upstream conditions seem unlikely to be the main reason for the large-amplitude motions of the separated bubble. 

in experiments, the characteristic ferquency of shock motions and separated bubbles are affected by the shock intensity and are not related directly to any time scale of the upstream boundary layer.
% ----

\section{Shock-Wave Boundary Layer Interaction}

here i'll talk about history, and experiments performed.




\subsection{History of SWBLI}

A history of both experimental and computational efforts are outlined. Dolling \cite{Dolling2001} provides a concise summary of 50 years of shock-wave/boundary layer interaction research up to 2001. Gaitonde \cite{Gaitonde2013} follows up with an update on the progress of this field of research up to 2013, especially with the advances of the computational tools in the last decade.

According to Dolling \cite{Dolling2001}, Ferri first recorded observation of SWBLI during testing of various airfoils at supersonic conditions \cite{Ferri1938, Ferri1940}. Shortly after, studies by Fage and Sargent, Ackeret et al., Liepmann \cite{Liepmann1946}, and Donaldson described the phenomenon at transonic speeds and the strong dependence of the interaction on the incoming boundary layer state. However, these early experiments were performed on curved surfaces, in a streamwise pressure gradient, and in small supersonic pockets within a subsonic flow which proved to be very difficult to investigate. Experiments in the late 1940s and late 1950s were performed in pure supersonic flow on canonical shapes still under investigation today: flat plates, external shock generators, flat plate with ramp, and flat plate with steps \cite{Beastall1950}, and axisymmetric bodies with flares or collars. 

In a general sense, as more sophisticated tools are developed, new opportunities arise to address questions that earlier workers may have raised but could not investigate, or to raise issues that were overlooked in earlier work. Often the new tools result in the discovery of new phenomena for investigation. This cyclic nature is quite evident in studies of shock-induced turbulent boundary-layer separation. From schlieren and shadowgraph photographs, investigators in the 1950s, including Chapman et al.,23 noted that such flowfields exhibited some degree of unsteadiness, but they did not have the means to study the issue further. Their focus was naturally on mean measurements, including wall pressures measured using conventional means (wall tappings), on surface flow visualization, with some measurements of heating rates, and very limited intrusive probing of the flowfield. As a consequence, key flow physics were overlooked. In the mid-1960s, when high-frequency response transducers and relatively high-speed data acquisition/recording became available, an opportunity arose to explore the unsteadiness. In one of the first, if not the first, studies of the phenomenon, Kistler26 found that shock- induced turbulent separation upstream of a forward-facing step was characterized by a relatively low-frequency, large-scale pulsation (low frequency relative to the incoming boundary-layer characteristic frequency $U_\infty/\delta$). This poses the question of how this could have been overlooked in earlier work. Chapman et al.23 used high-speed schlieren and shadow photography and noted that “turbulent separations were relatively steady.” They reported that “shock waves occasionally appeared to move slightly but no appreciable movement of the separated layer could be detected.”23 With hindsight, it is easy to see that their remarks on the relative steadiness of the turbulent cases were probably incorrect. In such two-dimensional flows, conventional spark shadow or schlieren photography will reveal only minor variations from one photograph to another, leading to the (erroneous) conclusion that a flowfield is essentially steady. However, unless the separation shock motion is uniform across the width of an interaction, the unsteadiness would not be detectable. These techniques average across the flowfield, and random span- wise variations in shock position result in essentially the same image from frame to frame. All that is revealed in such images is a slight rippling, suggesting that the flow is relatively steady. Thus, the free-interaction region in turbulent flows is actually characterized by a translating shock/compression system across which the instantaneous pressure rises abruptly. In the region downstream of X0, the mean pressure is the result of the superposition of large- amplitude shock-induced pressure fluctuations on the undisturbed boundary-layer pressure signal. The mean pressure increases downstream because of increasing shock frequency and fluctuation amplitude. Although the pressure can be scaled using the free-interaction parameters, the physics implicit in the latter are not what actually occurs.

A review of much of the work on flowfield unsteadiness up until the early 1990s can be found in Ref. 27. The majority of these unsteady measurements were restricted to surface pressures, from which only inferences about the external flowfield could be drawn. Some high-frequency response single-point (and sometimes multi- point) measurements have been made, but, due to flow interference, probes often cannot be placed where needed, or they cannot be made small enough. Furthermore, when intrusive measurements are made, the question of their validity always arises. Thus, although this body of work has brought out the basic character of the unsteadiness and has helped explain some features of the earlier mean flow measurements, it also generated a new set of questions regarding the causes of the unsteadiness. These remain unanswered today, and much work remains to be done in this area.

\subsection{Unsteadiness}

Unsteadiness: Perhaps the most significant breakthroughs in understanding the phenomenology and physics of SBLI since the publication of Ref. 10, have been in the area of unsteadiness. Numerous studies have explored the genesis of the prominent lowfrequency component inherent in SBLI, i.e., frequencies at levels well belowthe scales associated with turbulence in the boundary layer. This low frequency component is accompanied by separation shock foot movement and spanwise shock rippling, and has been documented to varying extent for all canonical configurations of Fig.1.28–32 A series of closely coordinated computational and experimental studies during the last decade, focused on nominally 2-D configurations, has provided plausible explanations for the observations. Results from these efforts, are summarized in Section III. The structural differences between 2-D and 3-D separation, specifically closed bubbles in the mean flow in the former versus open structures with little or no reversed flow in the latter as discussed in Section II, are then employed to comment on factors in extending the analysis to 3-D flows.
b. \cite{Gaitonde2013}

Several mechanisms have been proposed to explain the observations. As noted earlier, Ganapathisubramani et al58 observed that the incoming upstream boundary layer entering the interaction had long (streamwise) structures which they subsequently connected to similar motions in incompressible flows.75 These structures were also correlated to a waxing and waning of the boundary layer profile and could explain the unsteadiness since when the boundary layer was fuller/shallower, the shock moved downstream/upstream and further could also explain the observed spanwise variations because of their slender longitudinal character. On the other hand, Refs.76, 77 used their results to identify separation bubble dynamics including inherent instabilities of the shear layer, i.e., downstream dependence, as the dominant aspect. They showed that a physical model based on vortex shedding and entrainment, essentially an oscillator type of description, predicts theoretical time scales for the bubble breathing, which match observations at multiple Mach numbers and interaction strengths. All results captured peaks in a narrow range of 0.025 < StL < 0.04, where StL = fL/U∞. For reference, L varied depending on interaction strength from about 3.5 to 5.5 boundary layer thicknesses. \cite{Gaitonde2013}

\subsection{Flow Control}

The objective in general is increased efficiency or improved performance of a vehicle, or a component, hopefully at a lower cost, or with only a small (and worthwile) increment in cost. \cite{Dolling2001}. Kral77 notes that \say{the design trade-offs of a particular method of control must carefully be evaluated and compromises are often necessary to reach a particular design goal.} This makes flow control a particularly challenging endeavor because in many flows of interest it is not known with certainty what the key parameters are, let alone how changes in one parameter will affect others. \cite{Dolling2001}

Control: The traditional approach for SBLI control has been through bleed techniques, which alter the near wall low energy region to mute the deleterious effects of the interaction. The search for a more optimal solution has taken several directions, including passive (e.g., microramps with and without auxiliary bleed) and active (e.g., plasma-based or microjet) control techniques. The combination of a better understanding of the unsteadiness and the development of new high-bandwidth actuators has led to strategies that seek to leverage fluid instabilities with small disturbances of suitable spectral characteristics with more potential for scalability. Some key results are summarized in Section VI. \cite{Gaitonde2013}

\subsubsection{Vortex Generators}


\subsubsection{Bleed Holes}

The traditional approach for SBLI control has been through bleed techniques, which alter the near wall low energy region to mute the deleterious effects of the interaction. The search for a more optimal solution has taken several directions, including passive (e.g., microramps with and without auxiliary bleed) and active (e.g., plasma-based or microjet) control techniques. The combination of a better understanding of the unsteadiness and the development of new high-bandwidth actuators has led to strategies that seek to leverage fluid instabilities with small disturbances of suitable spectral characteristics with more potential for scalability. Some key results are summarized in Section VI. \cite{Gaitonde2013}

The traditional approach for SBLI control has centered around mass transfer. Bleed is frequently employed to alter the properties of the boundary layer, for example reducing its shape factor thus making it more robust to adverse pressure gradients, while blowing and transpiration are considerations for cooling purposes. The basic considerations in the use of mass transfer have been discussed in Ref. 4, which summarizes several issues such as optimal regions including location, distribution and inclination of holes or porosity. Simulation on a 3-D interaction due to a sharp fin with bleed and zero mass flux techniques may be found in Refs. 157 and 158 respectively. \cite{Gaitonde2013}

Active flow control is more advanced these days with pulsed plasma-jet actuators, plasma-based devices, and microjets. (written by me).

Control techniques based on this understanding have been attempted but may be considered to be in their infancy. \cite{Gaitonde2013}

Currently bleed is used to improve pressure recovery and mitigate flow distortion by reducing separation and unsteadiness. The ducting necessary for bleed incurs a drag, weight, and cost penalty. Furthermore, the inlet must be increased in size to account for the bleed massflow losses. An additional penalty of bleed at subsonic conditions is the associated aerodynamic roughness of the holes/slots. \cite{Dolling2001}

One method in which this unsteady shock can be controlled to produce a steady or predictable behavior might be through air suction using bleed holes. Bleed holes have traditionally been used to remove the lower momentum part of the boundary layer to delay transition and increase massflow through inlets \cite{Davis2012Fibe}. 

To better understand the flow structure involved in bleed, experimental research was conducted examining single holes \cite{Schoenenberger1999, Davis2012, Eichorn2013, Orkwis2013} and multiple holes \cite{Syberg1973, Paynter1993, Willis1995, Oorebeek2012}. Computational Fluid Dynamics (CFD) studies have followed \cite{Hamed2008, Choe2016, Duncan2016} that have provided additional insight and have also employed empirical curve fits as a bleed hole boundary condition \cite{Harloff1996, Baurle2011, Slater2012} in CFD codes. Through both experimental and computational results, bleed holes emerged as a stabilizing factor in flat plate SWBLI configurations\cite{Hamed1995}.

It is possible to address all of these concerns with mass-flow removal via boundary-layer bleed wherein one or more regions of the inlet wall are porous to remove mass from the boundary layer. Thus, boundary-layer bleed decreases the mass-flow to the engine, mitigates the effects of adverse pressure gradients, and reduces the boundary-layer thickness, and thus, the SWBLI. Bleed configurations often are characterized by the sonic flow coefficient (Q), which is defined as:

\subsection{Non-Dimensionalization}

%$ Q = \frac{w}{w^*_i} $

where the numerator is the measured mass-flow rate and the denominator is the ideal mass-flow rate under choked conditions:

%$ w^*_i = \frac{\qty(P_{t,e}\cdot A_b)}{\sqrt{T_{t,e}}} \cdot $

The sonic flow coefficient is typically presented as a function of the ratio of bleed plenum pressure to freestream (boundary-layer edge) total pressure (Pplen/Pt,e). For the present study, the total temperature at the boundary-layer edge (Tt,e) is assumed to be the same as the total temperature in the wind tunnel plenum chamber (Tt,0).
A large library of flow coefficient data has been developed, beginning with McLafferty (Ref. 1), which was then expanded by Willis (Ref. 2). These collections are very configuration specific as they are multi-hole or slot configurations that only cover a limited range of hole/slot geometries and bleed-hole orientations. Davis (Ref. 3) showed that with two adjacent 90° holes, the flow coefficient at a choked condition can vary by as much as 6 percent depending on hole orientation. While Slater (Ref. 4) has developed a model for 90° bleed holes based upon the Willis data, that model should be compared with single hole data where there is no interaction between adjacent holes. \cite{Eichorn2013}

The aerodynamic design of inlets for supersonic flight has commonly included the use of porous bleed regions to reduce the adverse effects of shock/boundary-layer interactions and to enhance the stability of the shock system (Refs. 1 to 10).

%Delery, J.M.: Shock Wave/Turbulent Boundary Layer Interaction and Its Control. Prog. Aerosp. Sci., vol. 22, no. 4, 1985, pp. 209-280.

%Seddon, J.: Intake Aerodynamics. AIAA Education Series, New York, NY, 1985.

%Hamed, A.; and Shang, J.S.: Survey of Validation Data Base for Shockwave Boundary-Layer Interactions in Supersonic Inlets. J. Propulsion, vol. 7, no. 4, 1991, pp. 617-625.

%Fukuda, M.K.; Hingst, W.R.; and Reshotko, E.: Bleed Effects on Shock/Boundary-Layer Interactions in Supersonic Mixed Compression Inlets. J. Aircraft, vol. 14, no. 2, 1977, pp. 151-156.

%Cubbison, R.W.; and Sanders, B.W.: Effect of Bleed-System Back Pressure and Porous Area on the Performance of an Axisymmetric Mixed-Compression Inlet at Mach 2.50. NASA TM X-1710, 1968.

%Shaw, R.J.; Wasserbauer, J.F.; and Neumann, H.E.: Boundary Layer Bleed System Study for a Full- Scale, Mixed-Compression Inlet With 45 Percent Internal Contraction. NASA TM X-3358, 1976.

%Mitchell, G.A.; and Sanders, B.W.: Increasing the Stable Operating Range of a Mach 2.5 Inlet. NASA TM X-52799, 1970.

%Mayer, David W.: Prediction of Supersonic Inlet Unstart Caused by Freestream Disturbances. AIAA J., vol. 33, no. 2, 1995, pp. 266-275.



 These bleed regions can consist of hundreds of small holes through which a portion of the low-momentum flow of the inlet boundary layer is extracted. This feature enhances the ability of the boundary layer to withstand the adverse pressure gradient and reduces the likelihood of boundary layer separation (Refs. 4 to 6). The bleed system can also help to remove excess flow to improve matching of flow rates between the inlet and engine (Refs. 9 and 10). This feature is important for stabilizing normal shocks near the throat of the inlet. The bleed flow is extracted by suction into a plenum and then either ducted for use by other aircraft systems or dumped overboard. While porous bleed has benefits, it requires a more complex and heavier inlet and can increase drag (Ref. 1 to 3). The effective use of porous bleed requires careful design of the location and flow rates for the bleed regions.

The methods of computational fluid dynamics (CFD) have been applied to the aerodynamic analysis of supersonic inlet flows containing bleed regions (Refs. 11 to 13). The small scale of the bleed holes results in the typical approach of modeling the effects of porous bleed through the use of surface boundary conditions. Various bleed boundary condition models have been reported by many researchers (Refs. 14 to 22). These models follow the general approach of assuming the bleed region to be a continuously porous surface. The solution points located within the bleed region are computed as boundary conditions in which the local bleed rates and velocity components are computed. The individual bleed holes are not identified nor are the details of the flow within the bleed holes computed. The models attempt to capture the collective behavior of the bleed holes.

The bleed model of Mayer and Paynter (Ref. 17) stands out as representing the current state of porous bleed modeling. This model was implemented within the Wind-US CFD code (Ref. 23). The inlet analyses of Reference 13 illustrate the use of this bleed model for a supersonic inlet analysis. The model allows the bleed rate to vary across the bleed region according to local conditions. This is important when shockwaves are interacting with the bleed region. For example, behind the shockwave, the static pressures are greater, which should result in a greater amount of bleed flow than ahead of the shock. The local bleed rate is calculated by extracting flow properties from the flow field and using a table lookup of empirically based sonic flow coefficients, Qsonic. The use of the Qsonicdata for the bleed model requires the CFD code to compute the Mach number, total pressure, and total temperature at the edge of the boundary layer. However, it may be computationally complex and time-consuming to locate each grid point at the boundary-layer edge, and can be especially difficult for unstructured-grid CFD codes. Furthermore, the edge of the boundary layer may not be well defined, such as in the case of a shock/boundary-layer interaction with extensive boundary-layer separation. Thus, a different approach for using the Qsonicdata is needed.

The current work improves on the Mayer-Paynter bleed model by introducing a scaling of the Qsonic data for 90° bleed holes. The scaling is able to collapse the Qsonicdata for various Mach numbers to a trend that can then be fitted with a quadratic polynomial, which is only a function of the ratio of plenum static pressure to the surface static pressure. The scaling eliminates the requirement to compute the flow properties at the edge of the boundary layer. The curve fit also provides a rudimentary model for blowing within a bleed region, which can occur if there is recirculation within the bleed region in the presence of a shock. The next section discusses the bleed modeling and the scaling of the Qsonicdata. The improved bleed model was implemented into the Wind-US CFD code.

The improved bleed model was demonstrated for flows over bleed regions on flat plates with and without shocks for which experimental data was available for comparison. The bleed model was also compared to three-dimensional CFD simulations of the flow through the bleed holes and plenum. Such simulations can provide details on the bleed flow useful for improving the bleed models. These simulations include a single bleed hole in uniform flow and a series of bleed holes interacting with an oblique shock. \cite{Slater2012}

