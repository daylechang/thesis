\chapter{Introduction} \label{introduction}
\section{Motivation}

Shock waves are a natural phenomenon that arise in many supersonic applications. When the shock interacts with the boundary layer, the situation becomes complex and is an important design consideration for a large number of applications in transonic and supersonic flows. This interaction occurs so pervasively that it has been termed the \abbreviationFull[shock-wave/boundary layer interaction]{SWBLI}. Successful applications of aircraft, missiles, inlets, or wind tunnel designs in supersonic flow is contingent on the favorable behavior of the SWBLI in both internal and external flows \cite{Gaitonde2013}. Three key issues that arise due to SWBLI are identified by Holden: peak heating, dynamic loads, and effects of separated unsteady flows \cite{Holden1986}. These three issues and their effects on practical applications are discussed in greater depth below followed by concluding remarks summarizing the motivation and objectives of this research.

%SWBLI is an important design factor for a large number of applications in transonic and supersonic flows. These interactions, however weak or strong, occur so ubiquitously that consideration must be given for successful application of any kind of transonic or supersonic application, whether the flow is internal or external and regardless of the body in quesiton, whether aircraft, missiles, rockets, and projectives. The SWBLI affects or determines the maximum, mean, and fluctuating pressure levels on a body, which in turn affect the local stresses on the body, heat flux, and flow structure which can have the same impacts on other body surfaces, though the overall energy levels are not affected very much. \cite{Gaitonde2013}

% (peak heating in three-dimensional turbulent interactions, dynamic loads generated in transitional regions of SWBLI, effects of the unsteadiness of separated flows, etc.) are as relevant today as research topics as they were 14 years ago. \cite{Dolling2001}]'

% - - - - - - - - - - - - - - - - - - - - - - - - - - - - - - - - - - - - - - - - - - - - - 
\subsection{Peak Heating}
The severe effects of peak heating is well documented in SWBLI, particularly in hypersonic flows. Peak rates vary between 10 to 100 times the incoming boundary-layer flow and many times the equivalent stagnation point value \cite{Dolling2001}. Knight and Degrez note that \enquote{heat transfer distribution predictions are generally poor, except for weak interactions, and significant differences are evident between turbulence models... of up to 100\% between experiment and numerical results} \cite{Knight1998}. These high temperatures cause localized stresses which affect the design of practical applications such as geometry, fatigue, material selection, and thermal protection \cite{Dolling2001}.

% - - - - - - - - - - - - - - - - - - - - - - - - - - - - - - - - - - - - - - - - - - - - - 
\subsection{Dynamic Loads}
In many aeronautical applications, parameters of critical importance are imposed by unsteady conditions that can occur during flight, rather than steady conditions. Although these events are rare or do not contribute much to the local average energy, they can correspond to high local stress, which can affect the whole behavior of the system. In supersonic flows, an important case occurs when unsteadiness involves shock waves producing locally large pressure fluctuations. They may act as strong aerodynamic loads and are felt along the whole flow downstream of the shock wave. This occurs in shock-induced separation, where low-frequency unsteadiness is produced \cite{Piponniau2009}.

An example of this is in the controls response system of aircraft and missile bodies. High-speed anti-armor, kinetic energy penetrators (Mach 4-8 at sea level) are configurations of interest to the U.S. Army that pose many simulation and experimental challenges due to the SWBLI. Following discharge from the gun, lateral thrusters or control surfaces may be used to guide and control such projectiles. Accurate characterization of the three-dimensional, unsteady, laminar, transitional, and turbulent interactions that sweep rapidly across is very important and will call for accurate modeling of SWBLI \cite{Dolling2001}. %\hl{also, hypersonics vehicles}

Another example is in plume-induced boundary-layer separation in missile design. The boundary layer separates on the afterbody of the missile rather than at the base itself, a result of the strong adverse pressure gradient generated by the interaction of the expanding plume and surrounding freestream ultimately creating large, unsteady, asymmetric loads on the body itself. The control surfaces and response system must have adequate control authority and response timing to overcome the unsteady loads \cite{Dolling2001}. In the worst case, the control surfaces themselves may experience premature boundary layer separation, causing a total loss of control effectiveness. In both cases, an understanding of the fundamental flow physics allows for missile designs to successfully demonstrate control authority during flight and exhibit maximum performance. Shaw et al. notes that this plume-induced separation feature is not unlike the SWBLI behavior on a compression ramp and share many common features, which can be used to help characterize the separation feature for missile applications \cite{Shaw1998}.

% - - - - - - - - - - - - - - - - - - - - - - - - - - - - - - - - - - - - - - - - - - - - - 
\subsection{Unsteady Flow}
Mixed compression inlets are designed to produce a shock train structure and terminal shock that allows for the highest total pressure recovery required for mixed compression inlets. However, these shocks are sensitive to flow disturbances and flow unsteadiness due to their interaction with the boundary layer. If the effects of unsteady flow are not mitigated, changes to the flow structure can displace the shock wave into a less efficient configuration. If the disturbances are large enough, the terminal shock may move upstream and ultimately out of the inlet, resulting in unstart that produces large transient forces on the airframe and cause engine surge \cite{Dolling2001}. % and ram/scramjet engines

Bleed holes are used as a method of active flow control to mitigate flow unsteadiness inside the inlet. % for both of these neutrally stable phenomenons (terminal shock and shock wave patterns). %\hl{insert compression figure}. 
By adjusting the mass-flow rate reaching the diffuser in response to perturbations in the engine operation or inlet flow, the shock train is stabilized. The disadvantage of such a method is that weight, energy, and thrust is reduced. Factors such as bleed hole location, hole geometry, suction massflow rate, and many others make this an area of ongoing research \cite{Dolling2001}.

% - - - - - - - - - - - - - - - - - - - - - - - - - - - - - - - - - - - - - - - - - - - - - 
\subsection{Current Views}
% B
SWBLI is recognized as a long-standing research area in the aerospace community, garnering wide attention both nationally and internationally \cite{Gaitonde2013}.

A 1996 NASA research announcement stated that \enquote{improved air-breathing engines will require a clearer understanding of the basic flow physics of propulsion system components.} The design of higher performance inlets and nozzles that are \enquote{quieter, shorter, lighter} requires \enquote{benchmark quality data for flowfields including shocks, boundary layers, boundary layer control, separation, heat transfer, surface cooling and jet mixing.} These areas all involve SWBLI in one form or another \cite{Dolling2001}.

The NASA \abbreviationFull[Small Business Innovation Research]{SBIR} solicitation in 2015 emphasized the need for basic research to be relevant for practical applications: \enquote{One of the greatest issues that NASA faces in transitioning advanced technologies into future aeronautics systems is the gap... between the maturity level of technologies developed through fundamental research and the maturity required for technologies to be infused into future air vehicles and operational systems} \cite{NASA2015}. The application of SWBLI research face difficulties on both fronts. On the fundamental research side, \enquote{inadequacies in our understanding of boundary layer turbulence increase reliance upon a more qualitative, physics-guided approach to discovery} \cite{NASA2008} whereas the application of SWBLI research \enquote{concepts for active and passive control of aeroacoustic noise sources... including adaptive flow control technologies, and noise control technology and methods that are enabled by advanced aircraft configurations, including integrated airframe-propulsion control methodologies} \cite{NASA2015}.

%\cite{Schmisseur2013}
%NASA continues to state in its \abbreviationFull[Small Business Innovation Research]{SBIR}
The \abbreviationFull[National Hypersonics Foundational Research Plan]{NHFSP} was developed by \abbreviationFull[Air Force Office of Scientific Research]{AFOSR}, \abbreviationFull[National Aeronautics and Space Administration]{NASA} and Sandia National Laboratory. This plan provides a framework of the main areas in hypersonics research of which SWBLI is a key component. The \abbreviationFull[Research and Technology Office, Air Vehicle Technology]{RTO/AVT} has been an integral part of that development under \abbreviationFull[North Atlantic Treaty Organization]{NATO} \cite{Schmisseur2013}.

Flow control for SWBLI remains a key issue for future technology vehicles but regarded as a complicated and vexing problem. To make the proper trade-offs in design, a deep, physical understanding of the mechanisms of these interactions must be understood with both experimental and computational tools that are both robust and accurate \cite{Dolling2001}.

% ---------------------------------------------------------------------------------
\section{Research Objectives} 

The advancement of \abbreviationFull[Computational Fluid Dynamics]{CFD} and its success shows promise as a critical tool in understanding SWBLI \cite{Zha1998, Dolling2001}. The objective of this research is to use CFD to demonstrate the effect of bleed holes on shock unsteadiness. This research will build further on canonical configurations such as ramps, swept-ramps, and fins by using a forward-facing step and determining if this configuration can be a successful model for mitigating shock unsteadiness.

This research will focus on characterizing the flow physics of a single bleed hole on a flat plate, characterizing the shock unsteadiness of a forward-facing step without active flow control, and observing the effects of bleed holes on the unsteady shock motion.

\Cref{introduction} introduced the subject, motivation, and objectives of this research. 
Chapter II%\Cref{background and theory} 
presents further background information on aerodynamics to include relevant research areas. In addition, it will provide fundamental theory for expected flow phenomena. 
Chapter III explains the computational methodologies used to characterize the computational setup including grid generation, flow solver parameters, turbulence modeling, and the overall experimentation approach. 
Chapter IV presents the results of the grid topology screening, time step/grid refinement study and aerodynamic characterization of the shock unsteadiness at supersonic conditions. The results will be analyzed and compared to experimental data and empirical models. A summary of the research along with conclusions and recommendations for future work are given in Chapter V.

% ---------------------------------------------------------------------------------
%NASA SBIR 2015 Phase I Solicitation
%A1 Air Vehicle Technology

% Inadequacies in our understanding of boundary layer turbulence increase reliance upon a more % qualitative, physics- guided approach to discovery. 
% 
% Concepts for active and passive control of aeroacoustic noise sources for conventional and % advanced aircraft configurations, including adaptive flow control technologies, and noise control % technology and methods that are enabled by advanced aircraft configurations, including integrated % airframe-propulsion control methodologies. 
% 
% 
% NASA SBIR 2015 Phase I Solicitation
% A2 Integrated Flight Systems
% 
% One of the greatest issues that NASA faces in transitioning advanced technologies into future % aeronautics systems is the gap caused by the difference between the maturity level of technologies % developed through fundamental research and the maturity required for technologies to be infused % into future air vehicles and operational systems.
% 
% NASA SBIR 2008 Phase I Solicitation
% A2 Fundamental Aeronautics
% 
% 
% Concepts for active and passive control of aero-acoustic noise sources for conventional and % advanced air- craft configurations, including adaptive flow control technologies, smart structures % for nozzles and inlets, and noise control technology and methods that are enabled by advanced % aircraft configurations, including advanced integrated airframe-propulsion control methodologies;
% 
% 
% This subtopic seeks innovative physics-based models and novel aerodynamic concepts, with an emphasis on flow control, applicable in part or over the entire speed regime from subsonic through hypersonic flight.



%Active flow control concepts targeted at separation control and/or viscous drag reduction with an %emphasis on the development of novel, practical, lightweight, low-energy actuators;
%
%Innovative methods to validate both flow models and aerodynamic concepts with an emphasis on aft-shock effects which are hindered by conventional wind tunnel model mounting approaches;


%B
%To limit these issues, configurations are over-designed with mitigation techniques active, passive, or both to increase control of the SWBLI, which quickly translate to detrimental effects on other important design parameters such as cost, development time, and weight, depending on the application. This directly contradicts the current aeronautical design trends such as commercial transport aircraft that aim to operate at increasingly higher efficiencies. SWBLI pose many challenges in supersonic flight, an area dominated by shock/boundary-layer interactions \cite{Dolling2001}.

% Harloff1996, Syberg1973, Schoenenberger1999, Saunders2014, Paynter1992} 
% Willis1995, Slater2009
% Papadakis2006, Orkwis2013, Oorebeek2012, Duncan2016
% 
% CFD
% duncan
% choe
% 
% Hamed1995
% Hamed2008
% 
% Experimental single hole
% davis
% orkwis
% *papadakis
% schoeneberger
% 
% eichorn2013
% 
% Experimental mult hole
% oorebeek
% paynter
% *saunders
% syberg
% willis
% 
% 
% CFD / mult hole
% harloff
% baurle
% slater



%\subsection{Abstract}
