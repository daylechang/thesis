% define custom commands
%\newcommand{\regmark}{\raisebox{5pt}{\tiny \circledR}\xspace}
%\newcommand{\trademark}{\raisebox{5pt}{\tiny TM}\xspace}
%\newcommand{\mca}{\texttt{Mathematica}\regmark}
%\newcommand{\Latex}{\LaTeX\xspace}

\newcommand*{\tensor}[1]{\overline{\overline{#1}}}
\newcommand*{\pO}{\partial\Omega}
\newcommand*{\dO}{\, d\Omega}
\newcommand*{\dS}{\, dS}
\newcommand*{\plus}{\texttt{+}}
\newcommand*{\minus}{\texttt{-}}
%\def\plus{\texttt{+}}
%\def\minus{\texttt{-}}
\newcommand*{\sfrac}[2]{\textstyle\frac{#1}{#2}}
	% style for superscript fractions
\newcommand*{\mdot}{\dot{m}}
\newcommand*{\trm}[1]{\textrm{#1}}
	% style for regular text in math mode

\DeclareMathAlphabet{\mathpzc}{OT1}{pzc}{m}{it} % for curly math

% Create a new theorem style called a Corollary.
% If you don't have any, then just comment this out.
%\theoremstyle{plain} % Default
%\newtheorem{cor}{Corollary}[chapter]

%Custom Commands for Student

\newcommand{\primerAddress}{{L:$\backslash$Courses$\backslash$PHYS$\backslash$LaTeX}\xspace}




 
% ----------------------------------------- CREF Package
\creflabelformat{equation}{#2\textup{#1}#3}
%\newcommand{\crefrangeconjunction}{\,\minus}
\newcommand{\crefrangeconjunction}{\,-}
	% make cross-references conjoined by -

%\crefname{equation}{equation}{equations}
%\Crefname{equation}{Equation}{Equations}% For beginning \Cref
%\crefrangelabelformat{equation}{(#3#1#4--#5#2#6)}

%\crefdefaultlabelformat{#2#1#3}

%\crefmultiformat{equation}{equations (#2#1#3}{, #2#1#3)}{#2#1#3}{#2#1#3}
%\Crefmultiformat{equation}{Equations (#2#1#3}{, #2#1#3)}{#2#1#3}{#2#1#3}